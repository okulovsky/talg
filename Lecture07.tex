\documentclass{beamer}

\usepackage{beamerthemesplit}

\usepackage[utf8]{inputenc}
\usepackage[russian]{babel}
\usepackage{amsmath,amsfonts,amsthm,amssymb,thmtools}

\newcommand{\eqnref}[1]{\eqref{#1}}
\newcommand{\secref}[1]{Section \ref{#1}}
\newcommand{\figref}[1]{Figure \ref{#1}}
\newcommand{\lemref}[1]{Lemma \ref{#1}}
\newcommand{\corref}[1]{Corollary \ref{#1}}
\newcommand{\thmref}[1]{Theorem \ref{#1}}
% Real numbers
\newcommand{\Real}[1]{\mathbb{R}^{#1}}
% Complex numbers
\newcommand{\Complex}[1]{\mathbb{C}^{#1}}
% Integers
\newcommand{\Integer}[1]{\mathbb{Z}^{#1}}
% Rank operator
\DeclareMathOperator{\rank}{\textnormal{rank}}
% Vec operator
\newcommand{\vecop}{\textnormal{vec}}
% Norm
\newcommand{\norm}[1]{\left|\left|#1\right|\right|}
% Trace
\newcommand{\trace}{\textnormal{tr}}
% Range
\newcommand{\range}{\textnormal{range}}
% Partial derivative
\newcommand{\pd}[2]{\dfrac{\partial #1}{\partial #2}}
% Complete derivative
\newcommand{\dd}[2]{\dfrac{d #1}{d #2}}
% Complete derivative, second order
\newcommand{\dds}[2]{\dfrac{d^2 #1}{d {#2}^2}}
% Limit to N / N
\newcommand{\limover}[1]{\lim_{#1 \rightarrow \infty} \dfrac{1}{#1}}
% Display style sum
\newcommand{\dsum}{\displaystyle\sum}
% arg min and arg max
\newcommand{\argmax}[1]{\underset{#1}{\operatorname{arg~max}}}
\newcommand{\argmin}[1]{\underset{#1}{\operatorname{arg~min}}}
%---------------------------------------------------------------------------
% Create theorem without number
\declaretheoremstyle[headfont=\bfseries,notefont=\bfseries,bodyfont=\itshape,notebraces={}{},headpunct={},postheadspace=1em]{mystyle}
\declaretheorem[style=mystyle,numbered=no,name=Теорема]{thm-hand}

\theoremstyle{plain}
\newtheorem{thm}{Теорема}
\newtheorem{lem}[thm]{Лемма}
\newtheorem{cor}[thm]{Следствие}
\newtheorem{conj}[thm]{Предположение}
\theoremstyle{definition}
\newtheorem{dfn}[thm]{Определение}
\newtheorem{Q}[thm]{Вопрос}
\newtheorem{prob}[thm]{Problem}
\newtheorem{prob-rus}[thm]{Задача}
\newtheorem{A}[thm]{Ответ}
\newtheorem{rmk}[thm]{Замечание}
\newtheorem{ex}[thm]{Пример}

\usetheme{Warsaw}

\title{Лекция 7}
\author{Подготовил Миронов Данил}

\begin{document}
\date{2.11.15}

\frame{\titlepage
\begin{center}
Теория алгоритмов 2015
\end{center}
}



\section{Задача}
\begin{frame}
	\frametitle{Promlem 1}
	\begin{prob}[1]
		По данным МТ: $M$ и $X$ определить $\exists$ ? $c$ : $|c|<p(|x|)$ : $M(x,c)=1$, где $p$ - известный полином \\
		При условии, что $M$ - ДМТ, работающая за полином \\
		В этом случае $P_1$ будет $NP$ полной задачей
	\end{prob}
	\begin{Q}
	    Почему эта задача в классе $NP$ ?	
	\end{Q}
\end{frame}

\begin{frame}
	\begin{Q}
	    Что такое класс $NP$ ?	
	\end{Q}
	\begin{A}
	    Класс задач, которые решаются на недетерминированнной МТ за полиномиальное время	
	\end{A}
	Определение плохое, так как НДМТ это не очень понятное свойство.\\
	Используем второе определение
\end{frame}

\begin{frame}
    \begin{dfn}
        $L$ - $NP$ язык, если $\exists$ $M_L$ - ДМТ, работающая за полином такая, что если некое слово $w\in L$, то $\exists c$, что $|c|\leqq p(|w|)$ и $M_L(w,c)=1$, где $c$ - сертификат
    \end{dfn}
    Это означает, что очень легко проверить, что некое слово принадлежит языку $L$. То есть если нам предъявят сертификат, то мы за полиномиальное время убедимся, что условие задачи лежит в $L$ 
\end{frame}

\begin{frame}
	\begin{ex}
	    Язык всех графов, содержащих гамильтонов цикл. \\
	    Тогда $w$ - какой-то граф \\
	    Говорим, что существует такая процедура сертификации, что граф будет принадлежать множеству всех графов, содержащих гамильтонов цикл, если существует некий сертификат полиномиальной длинны такой, что мой сертификат скажет - ОК \\
	    Таким сертификатом является собственно гамильтонов цикл
	\end{ex}
\end{frame}

\begin{frame}
	\begin{ex}[Продолжение]
	    Как мы понимаем, что гамильтонов цикл - это $NP$ язык ? \\
	    Мы знаем, что есть сертифицирующая машина, которая принимает граф и цикл. Затем убеждается, что этот цикл лежит в графе. \\
	    Дальше для каждого графа, который содержит гамильтонов цикл мы можем найти такой сертификат, что эта МТ скажет - ОК. \\
	    Если в графе есть гамильтонов цикл, то это будет сам гамильтонов цикл \\
	    Если нет гамильтонового цикла, то мы никак не сможем заставить эту машину сказать - ДА. \\
	    То есть такая сертифицирующая машина существует и поэтому утверждаем, что $L$ - $NP$ язык 
	\end{ex}
\end{frame}

\begin{frame}
	\begin{ex}[Итог]
	    Для $NP$ языков существует простая (полиномиальная) система проверки вхождения в этот язык на основании каких-то дополнительных данных.
	\end{ex}
\end{frame}

\begin{frame}
	\begin{Q}
	    Почему $P_1$ это $NP$ язык ?
	\end{Q}
\end{frame}

\begin{frame}
    Разбираемся по порядку
	\begin{Q}
	    Что здесь является словом языка ? \\
	    Что является сертификатом ?	
	\end{Q}

	\begin{A}
	    На входе есть $(M,x)$, нужно уметь проверять, что $(M,x)\in L_{p_1}$ \\
	    Язык $L_{p_1}$ - язык пар $(M,x)$, такой, что $\exists c$, такое, что $M(x,c)=1$, где $M$ - МТ, $c$ - некий вход \\
	    Cо всеми сопутствущими ограничениями на полиномиальность. \\
	    Сертификатом здесь будет $c$ \\
	    Так как в $L_{p_1}$ входит пара $(M,x)$ для которой $\exists c$ \\ Если нам предъявят $c$ то мы сможем убедиться, что пара в языке
	\end{A}
\end{frame}

\begin{frame}
	\begin{Q}
	    Какая у нас машина для процедуры сертификации ?
	\end{Q}
	\begin{A}
	    Машина, которая производит сертификацию вхождения слова в язык проблемы $P_1$ \\
	    У этой машины два аргумента: \\
	        - кандидат на слово языка $M_{L_p}$ \\
	        - сертификат $c$ \\
	  	Она должна отвечать ДА или НЕТ
	\end{A}
\end{frame}

\begin{frame}
	\begin{dfn}
	    Язык называется $NP$ полным, если \\
	    - лежит в классе $NP$ \\
	    - $NP$ трудный	
	\end{dfn}

	\begin{dfn}
		Язык $NP$ трудный, когда любая проблема из класса $NP$ сводится к нему по Карпу
	\end{dfn}
	
	\begin{dfn}
	    Сводимость по Карпу \\
	    $L_1\rightsquigarrow_c L_2$, если $\exists f$ такая, что $f(w)\in L_2 \Leftrightarrow w \in L_1$ \\
	    $f$ - полиномиальная функция, которая переводит слова, которые мы проверяем на принадлежность к $L_1$ в другие слова, которые мы проверяем на принадлежность к $L_2$ и образ слова принадлежит $L_2$ тогда и только тогда, когда само слово принадлежит $L_1$ 
	\end{dfn}
\end{frame}

\begin{frame}
	\begin{Q}
	    Почему $L_{p_1}$ $NP$ трудный ?
	\end{Q}
	\begin{A}
	    Пусть $L$ - какой-то $NP$ язык \\
	    Как $L$ свести по Карпу к языку $L_{p_1}$ \\
	    Нужно построить $f$ - полиномиальную функцию \\
	    $L$ - $NP$ язык ${\overset{\mbox{Опр}}\Rightarrow}$ $\exists M_L$ такая, что \\
	    - работает полиномиальное время \\
	    - $w \in L \Leftrightarrow \exists c : M_L(w,c)=1$ \\
	    $w$ превращаем в пару $(M_L, w)$ \\
	    Получаем формулировку $P_1$ задачи \\
	\end{A}
\end{frame}

\begin{frame}
	\begin{Q}
	    Как проверить, что $w \in L \Leftrightarrow (M_L,w) \in P_1$ ?	
	\end{Q}
	\begin{A}
	    $\Rightarrow$ \\
	    Известно, что $w \in L \Leftrightarrow \exists c: M_L(w,c)=1 \Rightarrow (M_L,w) \in P_1 \Rightarrow (M_L,w) \in L_{P_1}$\\
	    $\Leftarrow$ \\
	    Все переходы обратимы \\
	    $w \in L {\underset{\mbox{Опр $NP$}}\Leftrightarrow} \exists c: M_L(w,c)=1 {\underset{\mbox{Опр $P_1$}}\Leftrightarrow} (M_L,w) \in P_1 \Leftrightarrow (M_L,w) \in L_{P_1}$
	\end{A}
\end{frame}

\begin{frame}
	\begin{rmk}
	    Есть задачи, которые являются $NP$ полными и имеют естественную формулировку. 	
	\end{rmk}
\end{frame}

\begin{frame}
	\begin{prob-rus}[SAT (выполнимости)]
	    Язык булевых формул, которые удовлетворимы \linebreak
	    Пусть есть $x_1, \dots x_n$ - булевы переменные \linebreak
	    И набор дизъюнкций вида \linebreak
		$$
	    \begin{array}{l l}
	        	a_{11}(x_1) \vee a_{12}(x_2) & \vee \dots \vee a_{1n}(x_n) \\
	    	    &\vdots \\
	    	    a_{m1}(x_1) \vee a_{m2}(x_2) & \vee \dots \vee a_{mn}(x_n)
	    \end{array} 
		$$
	    Где 

	    \begin{center}
	    	$a_{ij} = \begin{cases} \neg & \\ \epsilon & \\ 0 & \end{cases}$
	    \end{center} 

	    Таким образом порождаем $m$ элементарных дизъюнктов
	\end{prob-rus}
\end{frame}

\begin{frame}
	\begin{rmk}
	    Дизъюнкты выполнимы $\Leftrightarrow \exists x_1 \dots x_n$, такие, что все они обращаются в истину
	\end{rmk}
	\begin{ex}
	    \begin{equation*}
	    	\begin{split}
	    		x_1 & \vee \neg x_2 \vee x_3 \\
	    		& \neg x_1 \vee \neg x_3 \\
	    		& x_1 \vee \neg x_2
	    	\end{split}
	    \end{equation*} \\
	    При $x_1=1$ $x_2=1$ $x_3=0$ получаем тожд. истинное высказывание, значит система выполнима
	\end{ex}
\end{frame}

\begin{frame}
	\begin{rmk}
	    Дизъюнкты выполнимы $\Leftrightarrow$ конъюнкция дизъюнктов равна $1$ при некоторых $x_1 \dots x_n$
	\end{rmk}
	\begin{rmk}
	    SAT - язык выполнимых КНФ
	\end{rmk}
	\begin{ex}[Невыполнимая КНФ]
	    \begin{equation*}
	    	\begin{split}
	    		& x_1\\
	    		\& & \\
	    		& \neg x_1 \vee \neg x_2 \\
	    		\& & \\
	    		& \neg x_2
	    	\end{split}
	    \end{equation*} \\
	\end{ex}
\end{frame}

\begin{frame}
	\begin{prob-rus}
	    Докажем, что SAT - $NP$ полный язык	(Теорема Кука)
	\end{prob-rus}
\end{frame}

\begin{frame}
	\begin{proof}
		$1)$ Покажем, что SAT лежит в $NP$ \\
		Нужно убедиться, что можно полиномиально быстро проверять решение. Здесь сертификат - это набор $x_1 \dots x_n$. Когда мы видим набор и видим формулу, мы можем сразу понять превращает ли он ее в истину.
	\end{proof}
\end{frame}

\begin{frame}
	\begin{proof}
		$2)$ Покажем, что каждая задача из $NP$ сводится к нашей \\
		Уже показали, что каждая задача из $NP$ сводится к $P_1$ \\
		То есть $\forall L \in$ $NP$ $L \rightsquigarrow_c P_1$ \\
		Сводимость по Карпу это транзитивное отношение \\
		(У нас будет композиция полиномиально вычислимых функций, которая полиномиально вычислима) \\
		Нужно доказать, что $P_1 \rightsquigarrow_c SAT$ \\
		Нужно для $(M,x)$ построить некую булеву формулу $B$ КНФ такую, что $B$ выполнима $\Leftrightarrow \exists c:M(x,c)=1$
	\end{proof}
\end{frame}

\begin{frame}
	\begin{proof}[Продолжение]
		Для доказательства приненим технику, которую мы использовали в доказательстве теоремы Геделя \\
		Сделаем булеву формулу, которая эмулирует работу МТ, поэтому она будет выполнима тогда, когда остановится исходная МТ. \\
		У нас будет тотальная, бинарная МТ \\
		$1.$ Бинарная: алфавит состоит из $1$ и $0$ $\Rightarrow x,c$ - битовые строки\\
		$2.$ Тотальная: для $\forall q \in Q$
		\begin{equation}
	    	\begin{split}
	    		q, 1 \Rightarrow q^1,s^1,d^1 \\
	    		q, 0 \Rightarrow q^0,s^0,d^0
	    	\end{split}
	    \end{equation}
	\end{proof}
\end{frame}

\begin{frame}
	\begin{rmk}
	    $*$ Для понятного объяснения далее считаем \\
	    Чтобы записать конкретные формулы в дальнейшем
	    \begin{equation*}
	    	\begin{split}
	    		q, 1 \Rightarrow q^1,0,\leftarrow \\
	    		q, 0 \Rightarrow q^0,s^0,.
	    	\end{split}
	    \end{equation*} \\
	\end{rmk}
\end{frame}

\begin{frame}
	\begin{proof}[Продолжение]
	    МТ бинарная, тотальная и полиномиально ограничена. \\
	    Ее память тоже ограничена полиномом \\
	    Если захотим хранить состояния машины за все время работы, то это все равно будет полином \\
	    Чтобы построить булеву функцию, определим, какие переменные будут в нее входить: \\
	    $M_{ij}$ - состояние $i$-ой ячейки памяти на такте $j$ \\
	    $Q_{ij}=1 \Leftrightarrow$ МТ находилась в состоянии $i$ на такте $j$ \\
	    $P_{ij}=1 \Leftrightarrow$ МТ на такте $j$ смотрела на позицию $i$
	\end{proof}
\end{frame}

\begin{frame}
	\begin{proof}[Продолжение]
	    Теперь хотим переписать \itshape{(1)} \normalfont{} как булеву функцию. \\
	    Берем состояние МТ $q$ и две инструкции, которые она выполняет, находясь в этом состоянии \\
	    Для каждого $j,i$ запишем формулы, где $i$ - индекс пробегающий ленту, $j$ - индекс, пробегающий время \\
	    $Q_{q,j} \& P_{i,j} \& M_{i,j} \rightarrow Q_{q^1,j+1} \& P_{i-1,j} \& \neg M_{i,j+1}$ \\
	    Время ограничено полиномом, память ограничена полиномом, количество состояний тоже ограничено полиномом, поэтому это все еще будет полином \\
	    Нужно еще добавить условия, чтобы это было осмысленно с точки зрения МТ: \\
	    - В каждый момент времени МТ может находиться только в одном состоянии \\
	    - В каждый момент времени МТ может смотреть только в одну ячейку памяти
	\end{proof}
\end{frame}

\begin{frame}
	\begin{proof}[Продолжение]
	    Сложная часть еще состоит в том, чтобы показать, что эти КНФ имеют полиномиальный размер. (Иногда, когда мы приводим к КНФ формула может приобрести полиномиальный размер) \\
	    \begin{center}
	    	''В данном случае верьте мне, они все будут полиномиальны''
	    \end{center}
	\end{proof}
\end{frame}

\begin{frame}
	\begin{proof}[Продолжение]
	    Как же с помощью построенного решить исходную задачу ? \\
	    $3.$ Если $M(x,c)=1$, то $\exists t$, что с момента времени $t$ $M$ находится в состоянии $\bar{q}$ \\
	    $4.$ Вход $M$ - это просто строка $xc$. Предполагаем, что есть какой-то внутренний формат, чтобы их разделить \\
	    Добавляем дополнительные формулы, так как хотим, чтобы полученная формула была не просто эмуляцией МТ, но и удовлетворяла начальному условию: \\
	    - $M_{i,0}=x_i, i<|x|$ \\
	    - $Q_{\bar{q}, D}$, где $D$ - общее количество тактов работы машины, которую мы эмулируем \\
	    Это все, так как свободными переменными останутся только переменные $M_{i,|x+1|}, \dots$
	\end{proof}
\end{frame}

\begin{frame}
	\begin{cor}
	    Ограниченную по времени МТ можно смоделировать одной булевой формулой 	
	\end{cor}
	\begin{cor}
	    Задача SAT - $NP$ - полна  	
	\end{cor}
\end{frame}

\begin{frame}
	\begin{rmk}
	    Многие задачи теории сложности формулируются на языке булевых формул	
	\end{rmk}

\end{frame}


\end{document}