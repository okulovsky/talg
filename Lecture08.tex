\documentclass{beamer}
\usepackage[T1,T2A]{fontenc}
\usepackage[utf8]{inputenc}
\usepackage[english,russian]{babel}
\usepackage{hyperref}
\hypersetup{unicode=true}


%Графы
\usepackage{tikz}
\usetikzlibrary{arrows}

\usepackage{tikz-qtree,tikz-qtree-compat}


\usepackage{tkz-graph}
\tikzset{
  EdgeStyle/.append style = {-} }
%Графы


%Стили для опеределений
\usepackage{amsthm}
\usepackage{amsmath}
\usepackage{amsfonts}
\theoremstyle{definition}
\newtheorem{defn}{Определение}
%Стили для опеределений



\usetheme{Singapore}

\title{Теория Алгоритмов}
\subtitle{Лекция 8}
\date{Екатеринбург, 2015}

\AtBeginSubsection[]
{
  \begin{frame}<beamer>{Оглавление}
    \tableofcontents[currentsection,currentsubsection]
  \end{frame}
}


\begin{document}

\begin{frame}
  \titlepage
\end{frame}

\begin{frame}{Оглавление}
  \tableofcontents
\end{frame}

\section{Лекция 7}
\subsection{Задача SAT}
\begin{frame}{Формулировка}
	\begin{defn}
		Проверить, является ли данная булева формула выполнимой
	\end{defn}
\end{frame}

\begin{frame}{Доказательство NP-полноты языка $L$}
	\begin{enumerate}
		\item $L \in NP$ 
		\item $L - NP$-труден
		\begin{enumerate}
			\item $N - NP$-труден
			\item $f(w) \in N \Leftrightarrow w \in L$
			\item $f$ - полиномиален
		\end{enumerate}
	\end{enumerate}
\end{frame}

\subsection{Задача 3SAT}
\begin{frame}{Формулировка}
	\begin{defn}
		Проверить, является ли данная булева формула, записаная в 3-конъюнктивной нормальной форме, выполнимой
	\end{defn}
\end{frame}


\begin{frame}{Докажем NP-полноту 3SAT}

	\begin{center}
	Имеем задачу:\\
	$(x_1 \lor \neg x_2 \lor x_3) \&$\\
	$(\neg x_1 \lor x_2 \lor x_3) \&$\\
	$(x_1 \lor x_2 \lor x_3)$
	\end{center}
	
	\begin{enumerate}
		\item 3SAT $\in NP$
		\item SAT $\leadsto\atop c$ 3SAT
	\end{enumerate}
		\begin{displaymath}
		\begin{matrix}\nonumber
  		a_{1,1} x_1 \lor a_{2,1} x_2  \lor & \cdots & a_{n,1} x_n\nonumber\\
  		\vdots & & \vdots\nonumber\\
  		a_{1,n} x_1 \lor a_{2,n} x_2  \lor & \cdots & a_{n,n} x_n\nonumber	
		\end{matrix}
		=
	\end{displaymath}
		
	
\end{frame}
\begin{frame}{}
	\begin{center}
		\textbf{Докажем $f(w) \in$ 3SAT $\Leftarrow w \in$ SAT}\\
		$w \in$ SAT $\Rightarrow$ хотя бы 1 из дизъюнктов равен 1 \\
		Имеем формулу $w$ из SAT:\\
		$(x_1 \lor x_2 \lor \neg x_3 \neg\lor x_4 \lor x_5)$\\
		Превратим ее в набор дизъюнктов $f(w)$ для 3SAT:\\
		$(x_1 \lor x_2 \lor y_1) \&$\\
		$(\neg y_1 \lor \neg x_3 \lor y_2)\&$\\
		$(\neg y_2 \lor \neg x_4 \lor x_5)$\\
		``Протаскиваем'' 1 так, чтобы $f(w) \Rightarrow w$\\
		
		\textbf{Докажем $f(w) \in$ 3SAT $\Rightarrow w \in$ SAT}\\
		Нельзя добиться истинности конъюнкций не используя ни одного $x=1$, а значит итоговая строка равна 1	
 	\end{center}	
\end{frame}

\subsection{Задача IS}
\begin{frame}{Формулировка}
	\begin{defn}
		По графу $G = (V,E)$ и числу $k$ узнать, существует ли независимое множество вершин $V'$, такое что $|V'|=k$
	\end{defn}
\end{frame}


\begin{frame}{Докажем NP-полноту IS}
	
	\begin{enumerate}
		\item IS $\in NP$
		\item IS - $NP$-трудна
		\item 3SAT $\leadsto\atop c$ IS
	\end{enumerate}
	$(x_1 \lor x_2 \lor \neg x_3) = F_1$\\
	$(x_1 \lor \neg  x_2 \lor x_3) = F_2$\\
	$(\neg x_1 \lor x_2 \lor x_3) = F_3$\\
	$(\neg x_1 \lor \neg x_3 \lor \neg x_2) = F_4$\\
	Каждую переменную в каждом дизъюнкте превращаем в вершину графа. Вершины соеденены сторонами если состоят в одном дизъюнкте (будем назыать их $E_1$), либо являются отрицанием друг друга $(x_i$~и~$\neg x_i)$) (будем называть их $E_2$)
\end{frame}

\begin{frame}{}
\begin{tikzpicture}
\Vertex[Math,L=x_1,x=0.9,y=1.5]{f1x1}
\Vertex[Math,L=x_2,x=1.3,y=0.3]{f1x2}
\Vertex[Math,L=\neg x_3,x=0.2,y=0.5]{f1nx3}

\Vertex[Math,L=x_1,x=4.8,y=0.3]{f2x1}
\Vertex[Math,L=\neg x_2,x=6,y=0.15]{f2nx2}
\Vertex[Math,L=x_3,x=7,y=0.5]{f2x3}

\Vertex[Math,L=\neg x_1,x=0.9,y=5]{f3nx1}
\Vertex[Math,L=x_2,x=1.5,y=5.8]{f3x2}
\Vertex[Math,L=x_3,x=0.5,y=6]{f3x3}

\Vertex[Math,L=\neg x_1,x=5,y=5.7]{f4nx1}
\Vertex[Math,L=\neg x_2,x=5.8,y=5]{f4nx2}
\Vertex[Math,L=\neg x_3,x=6,y=6]{f4nx3}
%Внутри дизъюнкта
\Edge(f1x1)(f1x2)
\Edge(f1x1)(f1nx3)
\Edge(f1x2)(f1nx3)

\Edge(f2x1)(f2nx2)
\Edge[style=bend left](f2x1)(f2x3)
\Edge(f2nx2)(f2x3)

\Edge(f3nx1)(f3x2)
\Edge(f3nx1)(f3x3)
\Edge(f3x2)(f3x3)

\Edge(f4nx1)(f4nx2)
\Edge(f4nx1)(f4nx3)
\Edge(f4nx2)(f4nx3)
%Между

\Edge(f1x1)(f3nx1)
\Edge(f1x1)(f4nx1)
\Edge(f2x1)(f3nx1)
\Edge(f2x1)(f4nx1)

\Edge[style=bend right](f1x2)(f2nx2)
\Edge(f1x2)(f4nx2)
\Edge(f3x2)(f2nx2)
\Edge(f3x2)(f4nx2)

\Edge[style={bend left=50}](f2x3)(f1nx3)
\Edge[style=bend right](f2x3)(f4nx3)
\Edge[style=bend right](f3x3)(f1nx3)
\Edge[style=bend left](f3x3)(f4nx3)
\end{tikzpicture}
\end{frame}

\begin{frame}{$w \in 3SAT \Rightarrow f(w) = (V,E,k) \in IS$}
	$\exists x_1 \ldots x_n :F_1(x_1 \ldots x_n) = F_2(x_1 \ldots x_n) = \ldots =1$\\
	Среди каждой тройки 3SAT существует формула, равная 1\\
	$E_1$ независисмы по определению \\
	$E_2$ независимы по построению $(x_i~\neq~\neg~x_i)$
\end{frame}

\begin{frame}{$w \in 3SAT \Leftarrow f(w) = (V,E,k) \in IS$}
	$\exists~IS~,V': |V'|=k$\\
	$x_i$ не может быть выбрано с $\neg x_i$\\
	Выбрано строго по 1 вершине в каждой тройке и нет противоречий, значит 3SAT выполнима.\\
	 
\end{frame}
\subsection{Известные NP-полные задачи}
\begin{frame}{Сводимость некоторых NP-полных задач}
\begin{tikzpicture}[sibling distance=9em,
  every node/.style = {shape=rectangle, rounded corners,
    draw, align=center,
    top color=white, bottom color=blue!20}]]
  \node {SAT}
    child { node {3SAT}
      child { node {3-сочетания}
        child { node {Разбиение\\множества}} }
      child { node {$k$ - раскраска\\графа} }
      child { node {Независимое\\множество}
        child { node {Clique}}
        child { node {Вершинное\\покрытие}
          child { node {Гамильтонов\\цикл}}}}};
\end{tikzpicture}
\end{frame}

\end{document}


