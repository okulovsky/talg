\documentclass[a4paper]{article}
\usepackage[utf8]{inputenc}
\usepackage[T2A]{fontenc}     
\usepackage[russian]{babel}
\usepackage[utf8]{inputenc}
\usepackage{bbold}
\usepackage{amsthm,amsmath,calc,tikz}
\newtheorem{theorem}{Утверждение}
\begin{document}
\section{Аналоговые машины }
{\large Аналоговые величины выражаются действительными числами.}
{{$\langle\mathbb Z  \cup  \{ \sqrt{2}, \sqrt{3}, \sqrt{4}\dots\}\rangle$}}
\\\\
{\large Операции:}
{\large
\begin{itemize}
  \item сложение
  \item умножение на константу
  \item применение нелинейной операции
\end{itemize} }
\begin{tikzpicture}[scale=1] % легко изменить масштаб
\draw[->]  (-1,0) -- (3,0) node[anchor=north] {$x$, сигнал};
\draw[->] (0,-0.5) -- (0,1.5) node[anchor=east] {$f(x)$};
\draw[green] (-1,0) -- (0,0) -- (1,1) -- (3,1);
\draw (-0.05,1) -- (0.05,1) node[anchor=west]{$1$};
\draw (1,-0.05) -- (1,0.05) node[anchor=north]{$1$};
\end{tikzpicture}
\section{Искуственная нейронная сеть}
\def\layersep{3cm}
{\large Модель, в которой можно посторить аналоговую машину.}
\\
{\large Нейрон: несколько входов, на каждом вес-константа.}
\\\\
\begin{tikzpicture}
\tikzstyle{big}=[draw, circle, fill=white, minimum size=40pt]
\node[draw=none,fill=none,anchor=east] at (0,-1) {$x_1$};
\node[draw=none,fill=none,anchor=east] at (0,-2) {$x_2$};
\node[draw=none,fill=none,anchor=east] at (0,-3) {$x_{n}$};
\node[big] (B-1) at (2,-2 cm) {$f(\Sigma w_{i}x_{i})$};
\path (0,-1) edge node [anchor=south] {$w_{1}$} (B-1);
\path (0,-2) edge node [anchor=south] {$w_{2}$} (B-1);
\path (0,-3) edge node [anchor=south] {$w_{n}$} (B-1);
\node[draw=none,fill=none] at (0,-2.5) {\vdots};
\end{tikzpicture}
\\\\
\begin{tikzpicture}[every node/.style=draw]
	\tikzstyle{neuron}=[circle, fill=white, minimum size=6pt]
	\\
	\node[draw=none,fill=none,anchor=east] at (15,-2) {\normalsize однонаправленная сеть нейронов без циклов};
	\node[neuron] (F-1) at (\layersep,-1 cm) {};
    \node[neuron] (F-2) at (\layersep, -2 cm) {};
    \node[neuron] (F-3) at (\layersep,-3 cm) {};
    \node[neuron] (S-1) at (\layersep+30, -1.5 cm) {};
    \node[neuron] (D-1) at (\layersep+60, -2.5 cm) {};
    \foreach \dest in {1,2}
		\foreach \name in {1}
			\path (F-\dest) edge (S-\name);
	\path (F-3) edge (D-1);
	\path (S-1) edge (D-1);
	\draw[->] (\layersep+65, -2.5 cm) -- (\layersep+95,-2.5 cm);

\end{tikzpicture}
\\\\
\begin{tikzpicture}
\node[draw=none,fill=none] at (0,0) {вход};
\tikzstyle{neuron}=[draw, circle, fill=white, minimum size=6pt]
\node[neuron] (L-1) at (0, -1){};
\node[neuron] (L-2) at (0, -2){};
\node[neuron] (L-3) at (0, -3){};
\node[draw=none,fill=none] at (0,-3.5) {\vdots};
\node[neuron] (L-4) at (0, -4.3){};
\node[draw=none,fill=none] at (5,0) {выход};
\node[neuron] (R-1) at (5, -1){};
\node[neuron] (R-2) at (5, -2){};
\node[neuron] (R-3) at (5, -3){};
\node[draw=none,fill=none] at (5,-3.5) {\vdots};
\node[neuron] (R-4) at (5, -4.3){};
\node[neuron] (M-1) at (2, -1){};
\node[draw=none,fill=none] at (3.5,-1) {\dots};
\path (L-1) edge (M-1);
\path (L-2) edge (M-1);
\path[->] (R-1) edge [out=15,in=165] (L-1);
\end{tikzpicture}
\\\\
{\large Цилиндрическая нейронная сеть с рациональными весами эквивалентна МТ.\\\\}
{\large$\Rightarrow$ Очевидно.\\}
{\large$\Leftarrow$ Идея на примере. Рассмотрим машину Минского.\\}
\begin{align*}
    q_{1} &: R_{1}++,\ q_{2} \\
    q_{2} &: R_{1}--\ ?\ q_{3}:q_{4}\\
    q_{3} &: R_{1}--\ ?\ q_{1}:q_{4}\\
    q_{4} &: STOP\\
\end{align*}
{\large Переводим ММ в архитектуру нейронных сетей.\\\\}
\begin{tikzpicture}
\node[draw=none,fill=none] at (0,0) {$q_{1}$};
\node[draw=none,fill=none] at (0,-1) {$q_{2}$};
\node[draw=none,fill=none] at (0,-2) {$q_{3}$};
\node[draw=none,fill=none] at (0,-3) {$q_{4}$};
\node[draw=none,fill=none] at (0,-4) {$R_1$};
\node[draw=none,fill=none] at (0,-5) {\vdots};
\node[draw=none,fill=none] at (0,-6) {$R_m$};
\tikzstyle{neuron}=[draw, circle, fill=white, minimum size=6pt]
\node[neuron] (L-1) at (0.5, 0){};
\node[neuron] (L-2) at (0.5, -1){};
\node[neuron] (L-3) at (0.5, -2){};
\node[neuron] (L-4) at (0.5, -3){};
\node[neuron] (L-5) at (0.5, -4){};
\node[neuron] (L-6) at (0.5, -6){};
\node[draw=none,fill=none,anchor=east] at (5,0) {$q'_{1}$};
\node[draw=none,fill=none,anchor=east] at (5,-1) {$q'_{2}$};
\node[draw=none,fill=none,anchor=east] at (5,-2) {$q'_{3}$};
\node[draw=none,fill=none,anchor=east] at (5,-3) {$q'_{4}$};
\node[draw=none,fill=none,anchor=east] at (5,-4) {$R'_1$};
\node[draw=none,fill=none,anchor=east] at (5,-6) {$R'_m$};
\node[neuron] (R-1) at (5.5, 0){};
\node[neuron] (R-2) at (5.5, -1){};
\node[neuron] (R-3) at (5.5, -2){};
\node[neuron] (R-4) at (5.5, -3){};
\node[neuron] (R-5) at (5.5, -4){};
\node[draw=none,fill=none] at (4.6,-5) {\vdots};
\node[neuron] (R-6) at (5.5, -6){};
\end{tikzpicture}
\\\\
{\large Кодируем регистры: $\varphi(R)=\frac{\varphi(R-1)+1}{2}$\\ $\varphi(0)=0 \ \Leftrightarrow \ R=0\\ \varphi(R) \in [\frac{1}{2};1), \ R>0 \\\\$}
{\large $q_1(t)=f(q_1(t-1),\dots,q_{n}(t-1),R_1(t-1),\dots,R_m(t-1))$\\ $q'_1=f(q_1,\dots,q_n,R_1,\dots,R_m)$\\ $q'_1=q_3\&(R_1>1)\\ q'_2=q_1\\q'_3=q_2\&(R_1>1)\\q'_4=[q_2\&(R_1=1)]\vee[q_3\&(R_1=1)]\\\\R'_1=\tilde{\&}(next(R_1),q_1)\oplus \tilde{\&}(prev(R_1),q_2)\oplus \tilde{\&}(prev(R_1),q_3)\oplus \tilde{\&}(R_1,q_4)$\\\\}
\\
{\large Введем функции:}
{\large
\begin{itemize}
  \item $next(x)=\frac{x+1}{2}$
  \item $prev(x)=2x-1$
  \item хитрая конъюнкция 
  \begin{equation*}
\tilde{\&}(x,y) = 
 \begin{cases}
   x, &\text{$y=1$}\\
   0, &\text{$y=0$}\\
 \end{cases}
\end{equation*}
\end{itemize}}
{\large Покажем как представить функции в нейронных сетях.\\\\}
\begin{tikzpicture}
\tikzstyle{big}=[draw, circle, fill=white, minimum size=40pt]
\node[draw=none,fill=none,anchor=east] at (0,0) {$ \& $};
\node[draw=none,fill=none,anchor=east] at (0,-1) {$1$};
\node[draw=none,fill=none,anchor=east] at (0,-2) {$x_1$};
\node[draw=none,fill=none,anchor=east] at (0,-3) {$x_{2}$};
\node[big] (B-1) at (2,-2 cm) {};
\path (0,-1) edge node [anchor=south] {$w_{0}$} (B-1);
\path (0,-2) edge node [anchor=south] {$w_{1}$} (B-1);
\path (0,-3) edge node [anchor=south east] {$w_{2}$} (B-1);
\end{tikzpicture}
\\\\
{\large $f(w_0+w_1x_1+w_2x_2)\\ w_0=0,w_1=1,w_2=1$\\}
\begin{equation*}
f(x,y) = 
 \begin{cases}
   1, &\text{$x>1$}\\
   0, &\text{$x<0$}\\
   x, &\text{$0<x<1$}
 \end{cases}
\end{equation*}
\\\\
\begin{tikzpicture}
\tikzstyle{big}=[draw, circle, fill=white, minimum size=40pt]
\node[draw=none,fill=none,anchor=east] at (0,0) {$ next(x) $};
\node[draw=none,fill=none,anchor=east] at (0,-1) {$1$};
\node[draw=none,fill=none,anchor=east] at (0,-3) {$R_1$};
\node[big] (B-1) at (2,-2 cm) {};
\path (0,-1) edge node [anchor=south] {$\frac{1}{2}$} (B-1);
\path (0,-3) edge node [anchor=south east] {$\frac{1}{2}$} (B-1);
\path (B-1) edge (4,-2);
\node[draw=none,fill=none,anchor=east] at (5,-2) {$\frac{R_1+1}{2}$};
\end{tikzpicture}
\\\\
\begin{tikzpicture}
\tikzstyle{big}=[draw, circle, fill=white, minimum size=40pt]
\node[draw=none,fill=none,anchor=east] at (0,0) {$ prev(x) $};
\node[draw=none,fill=none,anchor=east] at (0,-1) {$1$};
\node[draw=none,fill=none,anchor=east] at (0,-3) {$R_1$};
\node[big] (B-1) at (2,-2 cm) {};
\path (0,-1) edge node [anchor=south] {$-1$} (B-1);
\path (0,-3) edge node [anchor=south east] {$2$} (B-1);
\path (B-1) edge (4,-2);
\node[draw=none,fill=none,anchor=east] at (5.25,-2) {$2x-1$};
\end{tikzpicture}
\\\\
\begin{tikzpicture}
\tikzstyle{big}=[draw, circle, fill=white, minimum size=40pt]
\node[draw=none,fill=none,anchor=east] at (0,0) {$ \tilde{\&} $};
\node[draw=none,fill=none,anchor=east] at (0,-1) {$y$};
\node[draw=none,fill=none,anchor=east] at (0,-2) {$x$};
\node[draw=none,fill=none,anchor=east] at (0,-3) {$y$};
\node[big] (B-1) at (2,-2 cm) {};
\path (0,-1) edge node [anchor=south] {$-1$} (B-1);
\path (0,-2) edge node [anchor=south] {$1$} (B-1);
\path (0,-3) edge node [anchor=south east] {$1$} (B-1);
\end{tikzpicture}
\\\\
\begin{theorem}
{\large ЦНС с действительным весом может распознать любой язык.}
\end{theorem}
{$L=\{W_1,\dots,W_n,\dots\}$ - все МТ, которыве останавливаются на любом входе.\\ $\Sigma^*=\ '\ ','a','b',\dots,'aa','ab',\dots \\d=1,0,0,1,1,0,\dots \\ W={0,100110}_{2}\\ {[{0,100110}_{3}]_2}\ \ \ 0<x<1$\\\\}
{\large Строим сопроцессор для ЦНС, это тоже ЦНС, из которой будут выдаваться биты.\\\\}
\begin{tikzpicture}
\node[draw=none,fill=none] at (0,0) {$w$};
\node[draw=none,fill=none] at (0,-1) {$b$};
\tikzstyle{neuron}=[draw, circle, fill=white, minimum size=6pt]
\node[neuron] (L-1) at (0.5, 0){};
\node[neuron] (L-2) at (0.5, -1){};
\node[draw=none,fill=none,anchor=east] at (5.5,0) {$w'=2w-b'$};
\node[draw=none,fill=none,anchor=east] at (6.4,-1) {$b'=sign(2w-1)$};
\node[neuron] (R-1) at (2.5, 0){};
\node[neuron] (R-2) at (2.5,-1){};
\end{tikzpicture}
\\\\
{\large $w$-невычислимое число $\Rightarrow$ не существует алгоритма, который вычисляет с любой наперед заданной точностью.}
\end{document}