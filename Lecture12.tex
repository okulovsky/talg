\documentclass{beamer}

\usepackage{beamerthemesplit}

\usepackage[utf8]{inputenc}
\usepackage[russian]{babel}
\usepackage{amsmath,amsfonts,amsthm,amssymb,thmtools}

% Good footer
\addtobeamertemplate{navigation symbols}{}{%
    \usebeamerfont{footline}%
    \usebeamercolor[fg]{footline}%
    \hspace{1em}%
    \insertframenumber/\inserttotalframenumber
}

\newcommand{\eqnref}[1]{\eqref{#1}}
\newcommand{\secref}[1]{Section \ref{#1}}
\newcommand{\figref}[1]{Figure \ref{#1}}
\newcommand{\lemref}[1]{Lemma \ref{#1}}
\newcommand{\corref}[1]{Corollary \ref{#1}}
\newcommand{\thmref}[1]{Theorem \ref{#1}}
% Real numbers
\newcommand{\Real}[1]{\mathbb{R}^{#1}}
% Complex numbers
\newcommand{\Complex}[1]{\mathbb{C}^{#1}}
% Integers
\newcommand{\Integer}[1]{\mathbb{Z}^{#1}}
% Rank operator
\DeclareMathOperator{\rank}{\textnormal{rank}}
% Vec operator
\newcommand{\vecop}{\textnormal{vec}}
% Norm
\newcommand{\norm}[1]{\left|\left|#1\right|\right|}
% Trace
\newcommand{\trace}{\textnormal{tr}}
% Range
\newcommand{\range}{\textnormal{range}}
% Partial derivative
\newcommand{\pd}[2]{\dfrac{\partial #1}{\partial #2}}
% Complete derivative
\newcommand{\dd}[2]{\dfrac{d #1}{d #2}}
% Complete derivative, second order
\newcommand{\dds}[2]{\dfrac{d^2 #1}{d {#2}^2}}
% Limit to N / N
\newcommand{\limover}[1]{\lim_{#1 \rightarrow \infty} \dfrac{1}{#1}}
% Display style sum
\newcommand{\dsum}{\displaystyle\sum}
% arg min and arg max
\newcommand{\argmax}[1]{\underset{#1}{\operatorname{arg~max}}}
\newcommand{\argmin}[1]{\underset{#1}{\operatorname{arg~min}}}
%---------------------------------------------------------------------------
% Create theorem without number
\declaretheoremstyle[headfont=\bfseries,notefont=\bfseries,bodyfont=\itshape,notebraces={}{},headpunct={},postheadspace=1em]{mystyle}
\declaretheorem[style=mystyle,numbered=no,name=Теорема]{thm-hand}

\theoremstyle{plain}
\newtheorem{thm}{Теорема}
\newtheorem{lem}[thm]{Лемма}
\newtheorem{state}[thm]{Утверждение}
\newtheorem{cor}[thm]{Следствие}
\newtheorem{conj}[thm]{Предположение}
\newtheorem{rmk}[thm]{Замечание}
\newtheorem{proof-rus}[thm]{Доказательство}
\newtheorem{proof-rus-cont}[thm]{Продолжение доказательства}
\newtheorem{obs}[thm]{Наблюдение}
\newtheorem{dfn}[thm]{Определение}
\newtheorem{alg}[thm]{Алгоритм}
\theoremstyle{definition}
\newtheorem{Q}[thm]{Вопрос}
\newtheorem{A}[thm]{Ответ}
\newtheorem{prob-rus}[thm]{Задача}
\newtheorem{ex}[thm]{Пример}
\newtheorem{lsz}[thm]{Лемма Шварца-Зиппеля}

\usetheme{Warsaw}

\title{Лекция 12. Вероятностные компьютеры}

\begin{document}

\frame{\titlepage
\begin{center}
Теория алгоритмов 2015
\end{center}
}


\section{Лектор - Юрий Окуловский}
\begin{frame}
    \begin{dfn}
        Вероятностная машина Тьюринга (ВМТ) - недетерминированная машина Тьюринга, в которой переходы по веткам определяются вероятностым образом.
    \end{dfn}
    Сразу хочется ввести классы сложности, связанные с такими задачами. Например, обозначить задачи, которые можно посчитать на ВМТ за полином. Назовем этот класс $PP$ и определим следующим образом:
    \begin{dfn}
        $ L \subset PP : \exists M, w \in L \Leftrightarrow P(M(w)=1)>0.5$ и машина $M$ - полиномиальная.
    \end{dfn}
    То есть, мы берем слово, загружаем его в ВМТ, эта машина говорит нам "да" или "нет". Для тех слов которые лежат в этом языке ВМТ должна говорить "да" с вероятностью больше чем 0.5.
\end{frame}

\begin{frame}
    На самом деле это плохое определение, так как при таком определении $ NP\subset PP $. Чтобы это доказать, возьмем $NP$-полную задачу $SAT$ $F(x_1, ... , x_n) $ и воспользуемся следующим вероятностным алгоритмом $M_{sat}$:
    \begin{alg}
        Шаг1: $x_i=\left\{\begin{array}{l}1,\ p=1/2\\0,\ p=1/2\end{array}\right.$
        Шаг 2: Считаем F($x_i$) \\
        Шаг 3: if (F($x_i$) = 0) \\
            return $\left\{\begin{array}{l}1,\ p=1/2\\0,\ p=1/2\end{array}\right.$
        else return 1;
    \end{alg}
\end{frame}

\begin{frame}
    \begin{thm}
    $ NP \subset PP $
    \end{thm}
    \begin{proof-rus}
    Оценим вероятность того что $ w \not\subset SAT $. Так как мы будем всегда попадать в if $(F(x_i)$ = 0), то $P(M(w)=1)$ = 0.5 \\
    
    Допустим, что теперь $ w \subset SAT $.
    Пусть есть $k$ кортежей, которые удовлетворяют условию, тогда 
    рассмотрим следующие вероятности: \\
    $ P(M(v)=1$ | повезло) = 1  \\
    $ P $(повезло) = $\frac{k}{2^n} $ \\
    $ P(M(v)=1$ | не повезло) = 1 \\
    $ P $(не повезло) = 1 \\
    Используя формулу полной вероятности получаем: \\
    $ P(M(w)=1) = \frac{k}{2^n} + (1 - \frac{k}{2^n})\frac{1}{2}$ > 0.5 \\
    Это и доказывает, что $SAT$ лежит в $PP$
    \end {proof-rus}
    
\end{frame}

\begin{frame}
    \begin{rmk}
    Если алгоритм $M_{sat}$ выдает на выходе 1, это еще ничего не значит. Что бы это что-то значило, нужно провести статистически важное число испытаний. Другими словами, провести экспоненциальное число испытаний.
    \end{rmk}
    \begin{thm}
    $ PP \subset PSPACE $
    \end{thm}
    \begin{proof-rus}
    Нам надо показать: $ L \subset  PP \rightarrow L \subset  PSPACE $\\
    Возьмем какой нибудь язык $ L \subset PP \rightarrow \exists M, w \in L \Leftrightarrow P(M(w)=1)>0.5$ и машина $M$ - полиномиальная \\
    Хотим построить другую машину $M'$ - обычную ДМТ, которая вычисляет $ M'(w, M) = P(M(w)=1)$. Это можно сделать полным перебором с возвратом в классе $PSPACE$.
    \end{proof-rus}
\end{frame}

\begin{frame}
    \begin{proof-rus-cont}
    $M'$ - полиномиальна. Это означает, что количество ветвлений в любом случае будет не больше чем полином.
    Перебор в глубину будет не более чем полиномиальный $\rightarrow$ полиномиальной памяти хватит чтобы сохранить одно состояние перебора. $M'$ работает за полиномиальную память $\rightarrow$ $ L \subset PSPACE $
    \end{proof-rus-cont}
    \begin{rmk}
    В иерархии классов $PP$ лежит где-то между $NP$ и $PSPACE$
    \end{rmk}
\end{frame}

\begin{frame}
    Введем правильное определение
    \begin{dfn}
        $ L \subset BPP : \exists M, w \in L \Leftrightarrow P(M(w)=1)>\frac{2}{3}$ и машина $M$ - полиномиальная.
    \end{dfn}
    \begin{rmk}
        На самом деле вместо $\frac{2}{3}$ можно взять любое число больше $\frac{1}{2}$. Потому что при $k$ испытаниях вероятность ошибки изменится экспоненциально, а сложность запуска полиномиально.
    \end{rmk}
    Рассмотрим этот момент подробнее
\end{frame}

\begin{frame}
    Допустим у нас есть $P(M(w)=1) \geq  \frac{1}{2} + k$. Надо перейти к $P(M(w)=1) \geq \frac{2}{3}$. Рассмотрим на примере 2-ух монет. \\
    Есть 2 монеты: \\
    Первая: \\
        0.5 + p - веротятность выпадения орла \\
        0.5 - p - веротятность выпадения решки \\
    Соответствует случаю, когда $ w \subset L$ \\
    Вторая: \\
        0.5 - p - веротятность выпадения орла \\
        0.5 + p - веротятность выпадения решки \\
    Соответствует случаю, когда $ w \not\subset L$ \\
    Как понять с какой монетой мы имеем дело?
    \begin{rmk}
        Кидая достаточное количество раз монетку мы можем сделать веротяность ошибки сколь угодно маленькую. Вероятность ошибиться падает экспоненциально
    \end{rmk}
        
\end{frame}

\begin{frame}
    \begin{rmk}
        Возникает вопрос: $BPP = PP$?(например мы можем запустить алгоритм Монте-Карло, подождать полиномиально долгое число бросков и получить ответ). Поэтому возможно $BPP = PP$, но до сих пор не известно. Тем не менее многие задачи из $BPP$ со временем оказывались в $P$ (важный пример - проверка простоты).
    \end{rmk}
    Существует хороший пример задачи, про которую известно, что она лежит в $BPP$ и на сегодняшний день неизвестно лежит ли в $P$. Это задача о проверке полинома на тождественное равенство нулю.
\end{frame}

\begin{frame}
    \begin{prob-rus}
        $P(x_1, ..., x_n)$ - полином от n пемененных, задан в виде произведения полиномов или как определитель полиномиальной матрицы. \\
        Иначе говоря, полином задан таким образом, что мы не можем посчитать его коэффиценты(так как это будет полиномиальная задача). \\
        Задача: проверить равен ли полином $P$ нулю тождественно.
    \end{prob-rus}
    \begin{rmk}
        Никто пока не придумал детерминированного полиномиального алгоритма решения данной задачи.
    \end{rmk}
\end{frame}

\begin{frame}
    \begin{lsz}
        Пусть есть ненулевой полином $F(x_1, ..., x_n)$. $d$ - его степень. $S$ - конечное случайное подмножество $R^n$. $R$ - случайно выбранный элемент $S$. Тогда $P(F(R)=0) \leq \frac{d}{|S|}$ (маловероятно).
    \end{lsz}
    \begin{proof-rus}
        Будем доказывать индукцией по $n$. \\
        База индукции. $n=1$ \\
        $F(x_1)$ - полином от одной переменной. У него максимум $d$ корней. Сколько бы точек мы не взяли, максимум точек будет $d$ $\rightarrow$ $P(F(R)=0)=\frac{d}{|S|}$ и в частности $P(F(R)=0) \leq \frac{d}{|S|}$
        
    \end{proof-rus}
\end{frame}

\begin{frame}
    \begin{proof-rus-cont}
        Шаг индукции\\
        $F$ можно представить в виде:
        $F(x_1, ... , x_n) = \sum_{i=1}^d x_1^i \cdot F_i(x_2, ... , x_n)$ (берем $x_1$ и всюду выносим за скобки) \\
        $F \neq 0 \rightarrow$(из условия) $\exists i: F_i \neq 0$\\
        Выберем $i$ - максимальное из таких чисел: $i = max_{j} F_j \neq 0$.\\
        $Fi$ - ненулевой полином и $x_1^i$ тоже не нулевой. \\
        Выбираем случайное $R \subset S$ и оценим вероятность того, что $F(R) = 0$.
        
    \end{proof-rus-cont}
\end{frame}

\begin{frame}
    \begin{proof-rus-cont}
        Это может быть только в 2 случаях: когда $F_i = 0$ или когда $x_1^i = 0$\\
        1) $P(F_i(R)=0) \leq \frac{d-i}{|s|}$ (так как можно применить предположение индукции) \\
        2) $P(x_1^i(R) = 0) \leq \frac{i}{|s|}$\\
        Так как выполняется либо (1) либо (2) получаем:\\
        $P(F(R)=0)\leq \frac{d}{|s|}$
    \end{proof-rus-cont}
\end{frame}

\begin{frame}
    Про $BPP$ также известны следующие утверждения:
    \begin{rmk}
        $BPP \subset PP$ \\
        $BPP \subset \sum^2P \cap П^2P$ \\
        $P \cap BPP$
    \end{rmk}
\end{frame}

\end{document}