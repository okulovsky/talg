\documentclass{scrartcl}
\usepackage[utf8x]{inputenc}
\usepackage[russian]{babel}
\usepackage{amssymb}
\usepackage{tikz}
\usepackage{multirow}
\begin{document}
\title{Теория алгоритмов}
\subtitle{Лекция №3}
\author{Конспект Вячеслава Копейцева, КН-301\\Читает Юрий Окуловский}
\date{28.09.2015}
\maketitle
На прошлой лекции было закончено обсуждение машины Тьюринга (далее МТ) и дано определение машины Минского (далее ММ)
\section{Вычислительная мощность ММ}

Докажем, что ММ по вычислительным возможностям равномощна МТ т.е.:
\begin{enumerate}
\item $ MM \subseteq MT $
\item$ MT \subset MM $
\end{enumerate}
Доказательство:
\begin{enumerate}
\itemВозьмём МТ с $ \Sigma \{1, 0\} $
\itemДоговоримся, что $ \delta$ полностью определена
\end{enumerate}

$ q_i, 0 \to q_{ni}, s_{ni}, p_i^2$

$ q_i, 1 \to q_{yi}, s_{yi}, p_i^3$

Лента МТ:

\hspace*{18mm}$\nabla$

$\underbrace{1000} \underbrace{1} \underbrace{10}$

\ R1 \ \ R2 \ R3 $\leftarrow$ Регистры

\hspace*{18mm}$\uparrow$

Число, записанное в обратном порядке

ММ:

$p_i : R_2++ ; p_i'$

$p_i' : R_2--, p_i^2 : p_i^3$ - в зависимости от сравнения с 0

Рассмотрим состояния $ p_i^2$:

Нужно напечатать что-то на ленте

$ p_i^2$: если $S_{ni}$ = 0, то ничего;

\hspace*{15mm}$S_{ni}$ = 1, то $R_2$++;

В самой ММ условного перехода нет, т.к. всё уже определено

если $t_{ni} = \bullet$, то ничего;

если $t_{ni} = \leftarrow$, то: $R_3 = 2R_3 + R_2$

\hspace*{30mm}$R_2 = R_1 \% 2$

\hspace*{30mm}$R_1 = R_1 / 2$

\section{Тезис Чёрча-Тьюринга}
\begin{enumerate}
\item Любое мыслимое на сегоднящний день определение алгоритма будет эквивалентно МТ.
\item Всё, что мы можем вычислить, мы можем вычислить на МТ.
\end{enumerate}

\section{Теория вычислимости}
\subsection{Определения}

Опр. L - язык вычислимый (decidable) или разрешимый, если $\exists$МТ A, которая его распознаёт т.е.:

A : $\forall$ w $\in$ L A(w) = 1

\hspace*{6,5mm}$\forall$ w $\not\in$ L A(w) = 0

R - счётное число\\
Опр. Язык L - рекурсивно-перечислимый (semidecidable), если $\exists$МТ A такая что:

$\forall$ w $\in$ L A(w) = 1

$\forall$ w $\not\in$ L A(w) = МТ ничего не говорит.

Re - счётное число

R $\subset$ Re

Всего МТ - счётное число, т.к. текст имеет конечную длину т.е. |R| и |Re| - счётно.

L $\subseteq \Sigma^*$ 

$|\Sigma^*|$ = континуум $\Rightarrow \exists$ L $\not\in$ Re т.е. не распознаваемый МТ.

Опр. Язык L - кo-рекурсивно-перечислимый (CoRe), если $\exists$МТ A такая что:

$\forall$ w $\not\in$ L A(w) = 1

$\forall$ w $\in$ L A(w) = машина зацикливается.

\begin{figure}[h]
\centering
\def\seta{(0,0) circle (2.5)}
\def\setb{(-0.6,-0.8) circle (1.1)}
\def\setc{(0.6,-0.8) circle (1.1)}
\begin{tikzpicture}
     \tikzstyle{element}=[minimum size=1mm, font=\large]
    \node[element] ($all$) at (0,1) {Все языки};
    \node[element] ($Re$) at (-1,-1) {Re};
    \node[element] ($R$) at (0,-0.9) {R};
    \node[element] ($Core$) at (1,-1) {Core};
    \draw \seta;
    \draw \setb;
    \draw \setc;
\end{tikzpicture}
\end{figure}
Утв. R = Re $\cap$ Core.\\
Доказательство:\\
$\subseteq$:

L $\in$ R $\Rightarrow \exists$ MT A, которую можно разделить на две MT:
\begin{enumerate}
\item $A_1$ : $\forall$ w $\in$ L A(w) = 1

\hspace*{8mm}$\forall$ w $\not\in$ L A(w) = машина зацикливается.

L $\in$ Re
\item $A_0$ : $\forall$ w $\in$ L A(w) = машина зацикливается.

\hspace*{8mm}$\forall$ w $\not\in$ L A(w) = 1

L $\in$ CoRe
\end{enumerate}
$\supseteq$:

L $\in$ Re $\cap$ CoRe $\Rightarrow L \in Re, L \in CoRe$

Запустим параллельно:

\begin{tabular}{cc}
$A_1$ & \multirow{2}*{\huge\}\large A}\\
$A_0$ & \\
\end{tabular}\\
Утв. R замкнут относительно $\cup, \cap, \neg$ т.е.:

$\forall L_1, L_2 \in R : L_1 \cap L_2 \in R, L_1 \cup L_2 \in R, \Sigma^* \setminus L_1 \in R$\\
Доказательство:

Запустить обе MT и ждать результата их работы т.к. они обязательно должны остановиться.\\
Утв. Re замкнут относительно $\cup, \cap$ т.е.:

$\forall L_1, L_2 \in Re : L_1 \cap L_2 \in Re, L_1 \cup L_2 \in Re$\\
Доказательство:
Параллельно запускаем две МТ. В случае пересечения мы гарантировано за конечное время дождёмся остановки двух МТ, в случае объединения - одной.

\subsection{Теорема ''Проблема останова'' (Halting problem)}
Определить, останавливается ли MT A на входных данных X?

$L_H = \{''(A, X)''$ : A останавливается на X\}

$L_H \in Re$

$w \in L_H$

$w = (A, X)$

Покажем, что $L_H \not\in R.$

От противного:

\begin{tabular}{ll}
\multirow{2}*{$\exists MT M(A, X) = \{$} & 1, если A останавливается на X\\
& 0, если A не останавливается на X\\ 
\end{tabular}

Машина $M_1 : M_1(A) = M(A, A)$

\begin{tabular}{ll}
\multirow{2}*{$M_2(A) = \{$} & зацикливается, если $M_1(A)$ = 1\\
& 1, если $M_1(A)$ = 0\\ 
\end{tabular}

Рассмотрим чему равно $M_2(M_2)$:

Пусть $M_2(M_2)$ останавливается $\Rightarrow M(M_2, M_2) = 1 \Rightarrow M_1(M_2) = 1 \Rightarrow M_2(M_2)$ зациклится.

Если $M_2(M_2)$ зациклится $\Rightarrow M(M_2, M_2) = 0 \Rightarrow M_1(M_2) = 0 \Rightarrow M_2(M_2) = 1$ т.е. машина не зациклится.

Противоречие $\Rightarrow \not\exists M$.\\
Пример:

Рассмотрим код:
for x = 2,3...

\hspace*{36.5mm}y = 2,3...

\hspace*{36,5mm}n = 3,4...

\hspace*{36,5mm}z = 2,3...

\hspace*{36.5mm}if ($x^n+y^n==z^n$)

\hspace*{42.5mm}stop


С помощью такой МТ можно было бы доказать великую еорему Ферма, как и любую другую теорему из области конструктивной математики, но, к сожалению, очевидно, что такую МТ построить нельзя.
\end{document}
