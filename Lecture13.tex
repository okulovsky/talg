\documentclass{beamer}

\usepackage{beamerthemesplit}

\usepackage[utf8]{inputenc}
\usepackage[russian]{babel}
\usepackage{amsmath,amsfonts,amsthm,amssymb,thmtools}

% Good footer
\addtobeamertemplate{navigation symbols}{}{%
    \usebeamerfont{footline}%
    \usebeamercolor[fg]{footline}%
    \hspace{1em}%
    \insertframenumber/\inserttotalframenumber
}

\newcommand{\eqnref}[1]{\eqref{#1}}
\newcommand{\secref}[1]{Section \ref{#1}}
\newcommand{\figref}[1]{Figure \ref{#1}}
\newcommand{\lemref}[1]{Lemma \ref{#1}}
\newcommand{\corref}[1]{Corollary \ref{#1}}
\newcommand{\thmref}[1]{Theorem \ref{#1}}
% Real numbers
\newcommand{\Real}[1]{\mathbb{R}^{#1}}
% Complex numbers
\newcommand{\Complex}[1]{\mathbb{C}^{#1}}
% Integers
\newcommand{\Integer}[1]{\mathbb{Z}^{#1}}
% Rank operator
\DeclareMathOperator{\rank}{\textnormal{rank}}
% Vec operator
\newcommand{\vecop}{\textnormal{vec}}
% Norm
\newcommand{\norm}[1]{\left|\left|#1\right|\right|}
% Trace
\newcommand{\trace}{\textnormal{tr}}
% Range
\newcommand{\range}{\textnormal{range}}
% Partial derivative
\newcommand{\pd}[2]{\dfrac{\partial #1}{\partial #2}}
% Complete derivative
\newcommand{\dd}[2]{\dfrac{d #1}{d #2}}
% Complete derivative, second order
\newcommand{\dds}[2]{\dfrac{d^2 #1}{d {#2}^2}}
% Limit to N / N
\newcommand{\limover}[1]{\lim_{#1 \rightarrow \infty} \dfrac{1}{#1}}
% Display style sum
\newcommand{\dsum}{\displaystyle\sum}
% arg min and arg max
\newcommand{\argmax}[1]{\underset{#1}{\operatorname{arg~max}}}
\newcommand{\argmin}[1]{\underset{#1}{\operatorname{arg~min}}}
%---------------------------------------------------------------------------
% Create theorem without number
\declaretheoremstyle[headfont=\bfseries,notefont=\bfseries,bodyfont=\itshape,notebraces={}{},headpunct={},postheadspace=1em]{mystyle}
\declaretheorem[style=mystyle,numbered=no,name=Теорема]{thm-hand}

\theoremstyle{plain}
\newtheorem{thm}{Теорема}
\newtheorem{lem}[thm]{Лемма}
\newtheorem{state}[thm]{Утверждение}
\newtheorem{cor}[thm]{Следствие}
\newtheorem{conj}[thm]{Предположение}
\newtheorem{rmk}[thm]{Замечание}
\newtheorem{proof-rus}[thm]{Доказательство}
\newtheorem{proof-rus-cont}[thm]{Продолжение доказательства}
\newtheorem{obs}[thm]{Наблюдение}
\newtheorem{sled}[thm]{Следствие}
\newtheorem{dfn}[thm]{Определение}
\newtheorem{alg}[thm]{Алгоритм}
\theoremstyle{definition}
\newtheorem{Q}[thm]{Вопрос}
\newtheorem{A}[thm]{Ответ}
\newtheorem{prob-rus}[thm]{Задача}
\newtheorem{ex}[thm]{Пример}
\newtheorem{lsz}[thm]{Лемма Шварца-Зиппеля}

\usetheme{Warsaw}

\title{Лекция 13. Квантовые алгоритмы}
\date{14 Декабря, 2015}

\begin{document}

\frame{\titlepage
\begin{center}
Теория алгоритмов 2015
\end{center}
}


\section{Лектор - Юрий Окуловский}
\begin{frame}
    \begin{dfn}
        $\psi(x, y, z, r)$ - вероятность нахождения данной частицы в данном месте в данное время.
    \end{dfn}
    С помощью квантовых свойств можно построить машину, которая будет хранить кубиты. Кубиты могут находиться в трех состояниях: $1$, $0$ и неопределенность(пока не измерили).
    \begin{dfn}
        $\alpha | 0> + \beta|1>$ - запись состояния кубита. $\alpha$,$\beta$ - коэффициенты, причем $\alpha^2$, $\beta^2$ - вероятность измерения соответсвующего состояния.
    \end{dfn}
    \begin{sled}
        $\alpha^2 +\beta^2 = 1$
    \end{sled}
\end{frame}

\begin{frame}
    Далее запись вида $\textbf{\textit{2 | 0> + |1>}}$ подразумевает собой то, что на самом деле сущесвтует нормализирующий коэффициент, который принято опускать. В данном случае полная запись выглядит так: \\
    $\frac{1}{\sqrt[2]{5}}$ $(2 | 0> + | 1 >)$ \\
    \vspace{\baselineskip}
    Пример для записи состояния из двух кубитов:\\
    $(\alpha_1 | 0> + \beta_1 | 1>)*(\alpha_2 | 0> + \beta_2 | 1>)$
\end{frame}

\begin{frame}
    Однако вся соль квантовых алгоритмов заключается в том, что состояние квантовых систем из двух кубитов записывается таким образом:
    $$ \alpha_1 | 00> + \alpha_2 | 01> + \alpha_3 | 10> + \alpha_4 | 11> $$
    \begin{rmk}
        $$ \alpha_1^2 + \alpha_2^2 + \alpha_3^2 + \alpha_4^2 = 1 $$
    \end{rmk}
    \begin{rmk}
        Если система состоит из n кубитов, то имеем $2^n - 1$ степеней свободы. То есть мы получаем экспоненциальное пространство состояний достаточно маленькими усилиями.
    \end{rmk}
\end{frame}

\begin{frame}
    Возникает резонный вопрос - как пользоваться этими регистрами для вычислений? \\
    Нам нужно каким-то образом менять значения регистров. На самом деле существуют некоторые физические процессы, которые меняют коэффициенты регистров. \\
    Все эти процессы математические сводятся к следующему:
    $$ (\alpha_1, \alpha_2, ... , \alpha_n) * M = (\beta_1, \beta_2, ... , \beta_n) $$
    Где $M$ - некая матрица, а $ (\beta_1, \beta_2, ... , \beta_n) $ - новое состояние.
    \begin{rmk}
        1. $|M| = 1$ \\
        2. М - Эрмитова матрица
    \end{rmk}
\end{frame}

\begin{frame}
    \begin{ex}
        $$ f(x, y) = x \wedge y $$
        На самом деле функция $f(x, y)$ выглядит так: $$ f(x, y, z) = (x, y, x \wedge y ) $$
        Матрица должна быть размера $ 8*8 $: \\
        $$ M = 
             \begin{pmatrix}
              1 & 1 & 0 & 0 & 0 & 0 & 0 & 0 \\
              0 & 0 & 0 & 0 & 0 & 0 & 0 & 0 \\
              0 & 0 & 1 & 1 & 0 & 0 & 0 & 0 \\
              0 & 0 & 0 & 0 & 0 & 0 & 0 & 0 \\
              0 & 0 & 0 & 0 & 1 & 1 & 0 & 0 \\
              0 & 0 & 0 & 0 & 0 & 0 & 0 & 0 \\
              0 & 0 & 0 & 0 & 0 & 0 & 0 & 0 \\
              0 & 0 & 0 & 0 & 0 & 0 & 1 & 1
             \end{pmatrix}  $$
    \end{ex}
\end{frame}

\begin{frame}
    \begin{dfn}
        Матрица $M$ называется гейт.
    \end{dfn}
    \begin{dfn}
        Гейт Адамара:
         $$ \frac{1}{\sqrt[2]{2}} * 
             \begin{pmatrix}
              1 & 1 \\
              1 & -1
             \end{pmatrix}  $$
        $$ |0> \rightarrow |0> + |1> \rightarrow |0>$$
        $$ |1> \rightarrow |0> - |1> \rightarrow |1>$$
    \end{dfn}
    Попробуем определить, как действует гейт Адамара на такое произведение кубитов: $$ |x> = |X_1> * |X_2> * ... * |X_n>$$
\end{frame}

\begin{frame}
    После применения гейта Адамара на $X_1$ получим: $$ |X_1> \rightarrow |0> + (-1)^{X_1} | 1> $$
    Если всюду подставить эту формулу, то получим: $$ \sum_{y \in \{0, 1\}^n} -1^{x \star y} | y> $$
    Где $x \star y$ - скалярное произведение по модулю 2. \\
    Для более ясного понимания, разберем эту задачу при $ n = 3 $
\end{frame}

\begin{frame}
    $$ (|0> + (-1)^{X_1}|1>) * (|0> + (-1)^{X_2}|1>) * (|0> + (-1)^{X_3}|1>) $$
    Раскарыв скобки, мы получим элементы вида:
    $$ \begin{matrix}
      $|000>$ & $|100>$ \\
      $|001>$ & $|101>$ \\
      $|010>$ & $|110>$ \\
      $|011>$ & $|111>$
    \end{matrix} $$
    Перед каждым из них стоит множитель. Например $|011>$ умножается на $(-1)^0 * (-1)^{X_2} * (-1)^{X_3} $ и т.д. \\
    Этот множитель как раз и получается скалярным произведением булевых векторов.
\end{frame}

\begin{frame}
    \begin{prob-rus}
        \textbf{Задача Саймона}\\
        $f(x_1, x_2, ... , x_n) = (y_1, y_2, ... , y_n) $ - булева функция \\
        $f: \{0,1\}^n \rightarrow \{0,1\}^n$ \\
        $\exists a : f(x \bigoplus a) = f(x) \forall x$ - условие существования периода. \\
        Задача Саймона - найти период функции.
    \end{prob-rus}
    \begin{rmk}
        Эта задача имеет экспоненциальную сложность. Квантовые алгоритмы умеют делать это за линейное время.
    \end{rmk}
\end{frame}

\begin{frame}
    \textbf{Алгоритм Саймона}\\
    $\underbrace{|00..0>}_\text{n} \underbrace{|00..0>}_\text{m}$ \\
    1. $|00..0> |00..0>$ \\
    2. Применим гейт Адамара к первой части\\
        $\sum_{x \in \{0, 1\}^n} |x>|0>$ \\
    3. Теперь уже к обоим регистрам применяем функцию $f$. Другими словами применяем преобразование \\
        $ x, 0 \rightarrow x, f(x) $. Тогда получим\\
        $\sum_{x \in \{0, 1\}^n} |x>|f(x)> = \sum_{x \in \{0, 1\}^n}(|x> + |x + a>)|f(x)> $ \\
        Измерим последние $m$ кубитов.
    4. Измерение одного кубита влечет изменения состояния второго кубита. Это явление называется квантовой запутанностью.
        
\end{frame}

\begin{frame}
    После измерения мы будем знать что в первых n кубитах будет лежать:\\
    $\underbrace{(|x> + |x+a>)}_\text{n}\underbrace{f(x)}_\text{m}$ \\
    Применим гейт Адамара еще раз: \\
    \vspace{\baselineskip}
    $\sum_{y}(-1)^{x*y}|y> + \sum_{y}(-1)^{(x+a)*y}|y> =
        \sum_{y}[(-1)^{x*y} + (-1)^{x*y}*(-1)^{a*y}]|y>$ \\
    Пусть $a*y = 1$. Тогда получим: \\
    $(-1)^{x*y} + (-1)*(-1)^{x*y} = 0 \implies a*y = 1$ - событие нулевой вероятности. Тогда получим, что нам выпадают только такие $y$, что $ a*y = 0 $ \\
    Далее запускаем весь этот процесс несколько раз, чтобы получить $k$ независимых $y$.  Строим систему линейных уравнений над полем $Z_2$, решаем её методом Гаусса и получаем период $a$.\\
    \textbf{Конец алгоритма}
\end{frame}

\end{document}\end{document}
Status API Training Shop Blog About Pricing
© 2015 GitHub, Inc. Terms Privacy Security Contact Help