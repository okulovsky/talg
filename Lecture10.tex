\documentclass{beamer}

\usepackage{beamerthemesplit}
\usepackage{tikz}
\usepackage[utf8]{inputenc}
\usepackage[russian]{babel}
\usepackage{amsmath,amsfonts,amsthm,amssymb,thmtools}

% Good footer
\addtobeamertemplate{navigation symbols}{}{%
    \usebeamerfont{footline}%
    \usebeamercolor[fg]{footline}%
    \hspace{1em}%
    \insertframenumber/\inserttotalframenumber
}

\newcommand{\eqnref}[1]{\eqref{#1}}
\newcommand{\secref}[1]{Section \ref{#1}}
\newcommand{\figref}[1]{Figure \ref{#1}}
\newcommand{\lemref}[1]{Lemma \ref{#1}}
\newcommand{\corref}[1]{Corollary \ref{#1}}
\newcommand{\thmref}[1]{Theorem \ref{#1}}
% Real numbers
\newcommand{\Real}[1]{\mathbb{R}^{#1}}
% Complex numbers
\newcommand{\Complex}[1]{\mathbb{C}^{#1}}
% Integers
\newcommand{\Integer}[1]{\mathbb{Z}^{#1}}
% Rank operator
\DeclareMathOperator{\rank}{\textnormal{rank}}
% Vec operator
\newcommand{\vecop}{\textnormal{vec}}
% Norm
\newcommand{\norm}[1]{\left|\left|#1\right|\right|}
% Trace
\newcommand{\trace}{\textnormal{tr}}
% Range
\newcommand{\range}{\textnormal{range}}
% Partial derivative
\newcommand{\pd}[2]{\dfrac{\partial #1}{\partial #2}}
% Complete derivative
\newcommand{\dd}[2]{\dfrac{d #1}{d #2}}
% Complete derivative, second order
\newcommand{\dds}[2]{\dfrac{d^2 #1}{d {#2}^2}}
% Limit to N / N
\newcommand{\limover}[1]{\lim_{#1 \rightarrow \infty} \dfrac{1}{#1}}
% Display style sum
\newcommand{\dsum}{\displaystyle\sum}
% arg min and arg max
\newcommand{\argmax}[1]{\underset{#1}{\operatorname{arg~max}}}
\newcommand{\argmin}[1]{\underset{#1}{\operatorname{arg~min}}}
%---------------------------------------------------------------------------
% Create theorem without number
\declaretheoremstyle[headfont=\bfseries,notefont=\bfseries,bodyfont=\itshape,notebraces={}{},headpunct={},postheadspace=1em]{mystyle}
\declaretheorem[style=mystyle,numbered=no,name=Теорема]{thm-hand}

\theoremstyle{plain}
\newtheorem{thm}{Теорема}
\newtheorem{lem}[thm]{Лемма}
\newtheorem{state}[thm]{Утверждение}
\newtheorem{cor}[thm]{Следствие}
\newtheorem{conj}[thm]{Предположение}
\newtheorem{rmk}[thm]{Замечание}
\newtheorem{proof-rus}[thm]{Доказательство}
\newtheorem{obs}[thm]{Наблюдение}
\newtheorem{dfn}[thm]{Определение}
\theoremstyle{definition}
\newtheorem{Q}[thm]{Вопрос}
\newtheorem{A}[thm]{Ответ}
\newtheorem{prob-rus}[thm]{Задача}
\newtheorem{ex}[thm]{Пример}

\usetheme{Warsaw}

\title{Лекция 10}
\author{Подготовил Коврижных Алексей, КН-301}

\begin{document}
\date{23.11.15}

\frame{\titlepage
\begin{center}
Теория алгоритмов 2015
\end{center}
}


\section{Лектор - Юрий Окуловский}

\begin{frame}
    \frametitle{Язык задачи}
    \begin{dfn}
        $R = \{(x, y)\}$ - задача поиска. \\
        $L(R) = \{x | \exists y:(x,y) \in R\}$ - язык для задачи $R$ (условия, для которых есть решения)
    \end{dfn}
    Исследуем взаимосвязь задачи поиска с их языками.
    \begin{state}
        $R_1 \xrightarrow[l]{} R_2 \Rightarrow L(R_1) \xrightarrow[c]{} L(R_2)$
    \end{state}
    \begin{proof-rus}
        $R_1 \xrightarrow[l]{} R_2 \Rightarrow \exists f,g:(x, g(y)) \in R_1 \Leftrightarrow (f(x), y) \in R_2$ \\
        Нужно доказать, что $f(x) \in L(R_2) \Leftrightarrow x \in L(R_1)$ \\
        Пусть $f(x) \in L(R_2) \Leftrightarrow \exists y:(f(x),y) \in R_2 \Leftrightarrow (x, g(y)) \in R_1 \Leftrightarrow x \in L(R_1)$
    \end{proof-rus}
\end{frame}

\begin{frame}
    \begin{cor}
        $R \in FNP\text{-}C \Rightarrow L(R) \in NP\text{-}C$
    \end{cor}
    \begin{state}
        $\forall FNP\text{-}C R \xrightarrow[p]{} L(R)$
    \end{state}
    \begin{proof-rus}
        $R \xrightarrow[l]{} F\text{-}SAT \xrightarrow[p]{} SAT \xrightarrow[k]{} L(R)$
    \end{proof-rus}
    Сведение по Левину корректно, т.к. $F\text{-}SAT$ $FNP$-полна, сведение по Куку было доказано на прошлой лекции, сведение по Карпу корректно, т.к. L(R) - $NP$-полна.
    Теперь докажем, что все три сведения эквивалентны сведению по Куку.
\end{frame}

\begin{frame}
    \begin{proof-rus}
        Построим ДМТ (из определения сводимости по Куку) решающую задачу $R$. \\
        Алгоритм: \\
        1) По $x$ построить $F \in F\text{-}SAT$ \\
        2) Запустить алгоритм сведения по Куку $F\text{-}SAT \xrightarrow[p]{} SAT$ (см. предыдущую лекцию). Каждый раз, когда алгоритм вызывает оракул от формулы $Z$, заменяем $Z$ на $Z' \in L(R)$ с помощью преобразования (3) (по Карпу) и вызываем оракул $L(R).$
    \end{proof-rus}
\end{frame}

\begin{frame}
    \frametitle{Классы дополнений языков}
    \begin{dfn}
        $L \in CoNP \Leftrightarrow \overline{L} \in NP$
    \end{dfn}
    \begin{dfn}
        $L \in CoNP \Leftrightarrow \exists M_L \forall w \notin L \exists c : M_L(w,c) = 1$
    \end{dfn}
    Очевидно, что $\forall L : L \xrightarrow[p]{} \overline{L}$.
    \begin{state}
        $L \in NP\text{-}C \Rightarrow \overline{L} \in CoNP\text{-}C$
    \end{state}
    Для доказательства рассмотрим другое утверждение.
\end{frame}

\begin{frame}
    \begin{state}
        $L_1 \xrightarrow[k]{} L_2 \Rightarrow \overline{L_1} \xrightarrow[k]{} L_2$
    \end{state}
    \begin{proof-rus}
        $L_1 \xrightarrow[k]{} L_2 \Rightarrow \exists f : w \in L_1 \Leftrightarrow f(w) \in L_2 \Rightarrow w \notin L_1 \Leftrightarrow f(w) \notin L_2 \Rightarrow w \in \overline{L_1} \Leftrightarrow f(w) \in \overline{L_2}$
    \end{proof-rus}
    Докажем еще один простой факт:
    \begin{state}
        $P \subseteq NP \cap CoNP$
    \end{state}
\end{frame}

\begin{frame}
    \begin{proof-rus}
        $L \in P \Leftrightarrow \exists M\text{-ДМТ }\forall w$
            $M(w) = 0 \Leftrightarrow w \notin L$ \\
            \hspace{111px}$M(w) = 1 \Leftrightarrow w \in L$ \\
        Данное определение симметрично относительно отрицания, то есть $L \in P \Leftrightarrow \overline{L} \in P$, следовательно $P=CoP$. \\
        Тогда $\forall L : L \in P \Rightarrow \overline{L} \in P \Rightarrow \overline{L} \in NP \Rightarrow L \in CoNP$ \\
        Итак,
        \begin{enumerate}
            \item $P \subseteq NP$
            \item $P \subseteq CoNP$
        \end{enumerate}
        Отсюда следует, что $P \subseteq NP \cap CoNP$.
    \end{proof-rus}
\end{frame}

\begin{frame}

    % generated by GeoGebra with some fixes
    \begin{tikzpicture}[line cap=round,line join=round,x=0.7cm,y=0.7cm]
        \clip(-7.682766417784926,-2.021802239740338) rectangle (7.9123863666997485,6.908919490290925);
        \draw [rotate around={69.28738309242964:(0.7286617530349744,2.344494570353198)}] (0.7286617530349744,2.344494570353198) ellipse (3.0620206963167247cm and 1.6062195151831884cm);
        \draw [rotate around={-72.2087968912534:(-1.3068250598145614,2.3839283527795505)}] (-1.3068250598145614,2.3839283527795505) ellipse (3.0399890120776747cm and 1.5751833473383798cm);
        \draw(-0.35665573571062176,0.21965378120946383) circle (0.8800630605374081cm);
        \draw (-0.7381890337599272,0.59084161037458457) node[anchor=north west] {P};
        \draw [shift={(-2.720206214629994,6.687917209976597)}] plot[domain=4.239988859660625:5.8329927782917945,variable=\t]({1.*2.345629686975083*cos(\t r)+0.*2.345629686975083*sin(\t r)},{0.*2.345629686975083*cos(\t r)+1.*2.345629686975083*sin(\t r)});
        \draw [shift={(2.536056873768187,6.55019852643778)}] plot[domain=3.481634878871243:5.056428836473497,variable=\t]({1.*2.462561605935805*cos(\t r)+0.*2.462561605935805*sin(\t r)},{0.*2.462561605935805*cos(\t r)+1.*2.462561605935805*sin(\t r)});
        \draw (-3.00,6.00) node[anchor=north west] {NP};
        \draw (1.0287494590634596,6.00) node[anchor=north west] {CoNP};
        \draw (-1.4974680517285764,3.1779995975553343) node[anchor=north west] {FACTOR};
        \draw (-4.00,4.50) node[anchor=north west] {\parbox{1.8066458336802418 cm}{Изоморфизм \\ графов}};
        \begin{scriptsize}
        \draw [fill=black] (-0.3904651514316736,2.338302121542279) circle (2.5pt);
        \draw [fill=black] (-3.07336165389783,2.8756903218897096) circle (2.5pt);
        \end{scriptsize}
    \end{tikzpicture}

    \begin{dfn}
        $FACTOR$ - задача факторизации. По заданным $n$, $m$ определить, есть ли у числа $n$ простой делитель, больший чем $m$.
    \end{dfn}

\end{frame}

\begin{frame}
    \begin{thm}
    $FACTOR \in NP \cap CoNP$
    \end{thm}
    \begin{proof-rus}
        Возьмем сертификат $C=p_1^{k_1} \cdot p_2^{k_2} \cdot \dotso \cdot p_l^{k_l}$ = n (разложение числа на простые делители). По нему можно быстро (полиномиально) убедиться как в том, что у числа есть простые делители больше $m$, так и в обратном.
    \end{proof-rus}
    \begin{rmk}
        Существует полиномиальный алгоритм проверки на простоту - тест Агравала-Каяла-Саксены.
    \end{rmk}
\end{frame}

\begin{frame}
    \frametitle{Полиномиальная иерархия классов}
    \begin{itemize}
        \item $P$ - $M$-ДМТ, работающая за полином. $M(w)\in\{0, 1\}$
        \item $NP$ - $\exists ? x : M(w,x)=1$
        \item $CoNP$ - правда ли, что $\forall x M(w,x)=1$
        \item $FNP$ - найди $x : M(w, x)=1$
        \item $\Sigma^2P$ - $\exists ? x \forall y M(w,x,y)=1$
        \item $\prod^2P$ - правда ли, что $\forall x \forall y M(w,x,y)=1$
        \item $F\Sigma^2P$ - найди $x : \forall y M(w,x,y)=1$ (пример: Задача коммивояжёра)
    \end{itemize}
\end{frame}

\end{document}